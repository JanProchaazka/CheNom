\chapter{Dekompozice}
Protože je program rozsáhlý a jeden z~vytyčených cílů byla modularita, byl
rozčleněn na několik relativně (až na datové struktury a nutný interface)
nezávislých částí. Svou koncepcí se program blíží překladači.

Vstup je nejprve zpracován modulem \modul{Scanner} (lexikální analýza), který ho
rozdělí na tokeny a vrací jejich číselné kódy definované v~souboru
\soubor{terminaly.gra}. Navíc řeší základní kontextové problémy (vypouštění a vkládání
samohlásek), k~čemuž využívá soubor \soubor{konstanty.txt}. Tokeny vstupují jako
terminály do dalšího modulu zvaného \modul{Parser} (syntaktická analýza). Zde je
z~nich vystavěn derivační strom dle gramatiky v~souboru \soubor{gramatika.gra}.
V~tomto souboru navíc následuje za každým pravidlem sekvence instrukcí, které se
v~případě použití pravidla aplikují v~příslušném uzlu. To zajistí modul
\modul{Semantika} (sémantická analýza). Výstupem tohoto modulu je již graf
molekuly. Ten je třeba umístit do prostoru, aby vznikl model, což zajišťuje
modul \modul{Model}, který pomocí volání modulu \modul{Grafika} zajistí i výstup
ve for\-má\-tu VRML \cite{Zrzavy:vrml} s~parametry definovanými v~souborech
\soubor{prvky.gra} (zde jsou veškeré potřebné parametry jednotlivých prvků --
náboj, váha, přípustné elektronové konfigurace, poloměr a barva) a
\soubor{sablona.wrl} (zde je uložena celá hlavička výsledného dokumentu, to
znamená Viewpoints, uzly PROTO pro atomy a vazby, zvolené kódování, atd.).

Jména všech použitých souborů jsou definována v~souboru \soubor{settings.ini},
který program dostane jako parametr.

Pro větší uživatelovo pohodlí byla implementována dvě rozhraní. Konzolový skript
\soubor{run-console.bat} (popř. \soubor{run-console.sh}) přijímá název molekuly
jako parametr a vytvořený VRML model zobrazí v~aplikaci asociované s~příponou
,,wrl". Druhý interface pak tvoří jednoduchý HTTP server, který přijímá dotazy
na stránku s~modelem molekuly, poté zavolá skript \soubor{run.bat}
(popř. \soubor{run.sh})
s~názvem v~parametru. Pokud takový ještě nebyl vytvořen, tak ho vytvoří (pomocí
volání výše popsaného jádra) a vrátí stránku, která model obsahuje.

Z~výše uvedeného je zřejmé, že program potřebuje pro svou plnohodnotnou práci
buď plugin pro zobrazování VRML do nějakého internetového prohlížeče (v~případě
užití druhého interface) nebo jakýkoliv jiný program schopný interpretace VRML
souborů (v~případě užití prvního -- ale v~podstatě i druhého -- interface).

\begin{figure}
	\imgbox{dekompozice}
	\caption{Schéma zapojení programových modulů a jejich komunikace}
\end{figure}
