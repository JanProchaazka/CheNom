\begin{titlepage}
\begin{center}
\large
Univerzita Karlova v~Praze\\
Matematicko-fyzikální fakulta\\
\vspace{35mm}
%\normalsize
{\Large\bf Vizualizace molekul \\ popsaných v~ženevském názvosloví}\\
\vspace{5mm}
{\Large Jan Procházka}\\
\vspace{15mm}

%\includegraphics[scale=0.3]{grafika/all/pdf/logo.pdf}\\
\end{center}
\vfill

{
\small 
\noindent
Součástí této práce je i CD, jehož obsah je možné stáhnout na adrese: \\
http://hippies.matfyz.info/projekty/chenom
}
\vspace{5mm}
\begin{center}
SVOČ, 2008
\end{center}

\end{titlepage} % zde končí úvodní strana

\normalsize % nastavení normální velikosti fontu
\setcounter{page}{2} % nastavení číslování stránek
\vspace{\fill}

\tableofcontents % vkládá automaticky generovaný obsah dokumentu
%\newpage % přechod na novou stránku
%
%%% Následuje strana s abstrakty. Doplňte vlastní údaje.
%\noindent
%Název práce: Vizualizace molekul popsaných v~ženevském názvosloví\\
%Autor: Jan Procházka\\
%Katedra: Katedra aplikované matematiky\\
%Vedoucí bakalářské práce: RNDr. Martin Pergel, KAM\\
%e-mail vedoucího: perm@kam.mff.cuni.cz\\
%
%\noindent Abstrakt:
%Předmětem této práce je implementace nástroje na automatické zpracování názvů
%organických sloučenin do podoby prostorového modelu molekuly. Pro popis modelu
%je použit jazyk VRML, což umožňuje mimo jiné následnou práci s~ním ve většině
%nástrojů určených pro práci s~prostorovými objekty. Program lze ovládat buďto
%z~příkazové řádky, nebo přes WWW-formulář. Tím je dosaženo jak komfortu při
%běžném používání programu, tak možnosti jeho využití jinými programy. Důraz byl
%kladen na zpracování jazyka názvosloví organické chemie a vytvořený model není
%přesný, avšak pro získání představy o struktuře dostatečný. \\
%
%\noindent Klíčová slova: chemie, názvosloví, modelování
%
%\vspace{10mm}
%
%\noindent
%Title: Visualization of Molecules Described in Geneva System\\
%Author: Jan Procházka\\
%Department: Department of applied mathematics\\
%Supervisor: RNDr. Martin Pergel, KAM\\
%Supervisor's e-mail address: perm@kam.ms.mff.cuni.cz\\
%
%\noindent Abstract:
%The subject of this work is to implement the tool for automatic processing of
%the names of the organic compounds into three-dimensional models of the
%molecules. The VRML language is used for description of the model. It allows
%(apart from other things) next processing in most tools designated for work with
%three-dimensional subject. It is possible to control the program from the
%command line or over the WWW-form. It causes both comfort during common use and
%possibility of next using by other programs. We focused on processing of the
%language of the organic chemistry, while the created model is not precise.
%However, for acquisition of general information about the structure it suffices.
%\\
%
%\noindent Keywords: chemistry, nomanclature, modeling
%
