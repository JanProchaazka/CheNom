\chapter{Závěr}
Cílem této práce bylo vytvořit program, který by pomohl především studentům
středních škol při studiu názvosloví organické chemie. Tento vytyčený cíl byl
snad programem CheNom naplněn. Přesto je možné mnohé věci vy\-lep\-šit.

\section{Možnosti rozšiřování}
Určitá zlepšení je možné provést zpřesněním modelování, rozšířením o~interpretaci
názvů některých druhů sloučenin a stereochemických názvů nebo zjednodušením
instalace.

Nepřesnosti v~modelování spočívají v~přílišném zjednodušení použitého modelu.
Bylo by ji vhodné nahrait za některou~chemiky používanou metodu, jako například
hybridizace nebo VSEPR\cite{Micka:anorgana}. Systematické názvosloví heterocyklů
a kondenzovaných polycyklů nebylo implementováno, protože má zcela odlišná
pravidla tvoření názvů a často jsou tyto názvy chápány spíše jako triviální.
Implementovat stereochemická pravidla nemá smysl, dokud je použit pro modelování
současný model, který není schopen zaručit jejich dodržení.

Dalšího zpřesnění modelu by bylo dosaženo, kdyby byl program schopen při dalším
dotazu na stejnou molekulu ji dále aproximovat (např. vždy po dobu dalších 10\,s).

Pro větší pohodlí uživatele by bylo možné vytvořit instalátor / instalační
skript, který by umožňoval nakonfigurovat program automaticky. Jednalo by se o
umístění zástupce programu na plochu či do nabídky start, určení používaného
interface, rozsahu cachování, atd.

\section{Alternativní implementace}
V~průběhu tvorby programu byly zvažovány různé způsoby implementace jednotlivých
modulů.

V~modulu \modul{Parser} přicházela v~úvahu alternativa urychlující proces stavby
derivačního stromu. Program by neprocházel stále dokola všechna pravidla
gramatiky, ale místo toho použil zásobník (nebo frontu) na uložení těch
pravidel, která by bylo možné použít. Seznam pravidel v~zásobníku (frontě)
nejprve tvoří všechna pravidla gramatiky. Při snaze o vytvoření derivačního
stromu jsou nepoužitá pravidla odstraňována a naopak jsou přidávána pravidla
po\-u\-ži\-tá pravidla obsahující. Gramatika programu CheNom ovšem není natolik
rozsáhlá, aby takto pracná implementace měla význam (doba výpočtu v~modulu
\modul{Model} je totiž výrazně delší).

V~modulu \modul{Server} byla zvažována možnost vkládání různých souborů
(obrázky, skripty, kaskádové styly, atd.) do WWW stránky rozhraní. Tato možnost
byla z~bezpečnostních důvodů zavrhnuta. Bylo by nutné ošetřit, aby uživatel
nemohl program zneužít pro obejití nepovoleného přístupu k~souborům.

\section{Porovnání s~existujícím SW}
V~současnosti je na trhu nepřeberné množství programů zaměřených na problematiku
různých oblastí chemie. Co se týče interpretace českého názvu organických
sloučenin, žádný takový software neexistuje.

Příkladem programu, který se také zabývá vizualizací molekul, je program
ACD/Labs~8.0. Jeho předností je přesný model molekuly, ve kterém je možné také
měřit vzdálenosti mezi atomy a vazebné úhly. Navíc umožňuje snadno přecházet
mezi různými typy zobrazení (např. barvy a velikosti atomů). Naopak nevýhodami
jsou možnost práce s~názvem pouze ve směru od struktury, neschopnost programu
v~modelu graficky rozlišit násobnosti vazeb a jeho nedostupnost v~češtině (a
tedy ani programem generované názvy nejsou česky).

