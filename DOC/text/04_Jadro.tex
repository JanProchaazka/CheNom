\chapter{Technická realizace -- jádro}
Tato kapitola obsahuje podrobný rozbor implementace modulů, ze kterých se skládá
jádro programu, a jejich vzájemné komunikace. Těmito moduly jsou
\modul{Scanner}, \modul{Parser}, \modul{Semantika}, \modul{Model} a
\modul{Grafika}.

%%%%%%%%%%%%%%%%%%%%%%%%%%%%%%%%%%%%%%%%%%%%%%%%%%%%%%%%%%%%%%%%%%%%%%%%%%%%%%%%
\section{\modul{Scanner}}
Jak již bylo řečeno, tento modul má za úkol provést lexikální analýzu
po\-ža\-do\-va\-né\-ho názvu. To není vhodné učinit standardním způsobem, totiž po\-sta\-ve\-ním
automatu, který je resetován po každém přijatém tokenu, protože toto
předzpracování by bylo (vzhledem k~počtu tokenů, které budou rozpoznány) příliš
časově náročné.

Bylo tedy použito jednodušší řešení. Program postaví z~tokenů načtených ze
souboru \soubor{terminaly.gra} lexikální strom. Při zpracovávání názvu se v~něm hledá
vždy nejdelší cesta od kořene odpovídající dosud nepřečtenému tokenu. Při
dosažení uzlu ve stromě, který již nemá přechod přes následující znak v~názvu či
při dosažení konce názvu se program vrátí zpět do posledního uzlu, který je
označen jako výstupní. Z~odpovídajícího místa v~názvu pak pokračuje nové hledání.
Pokud se stane, že je při návratu po větvi dosažen kořen, znamená to, že se
zbývající část názvu nepodařilo určit. Program se vrátí do posledního určeného
tokenu a pokračuje se v~hledání (směrem ke kořenu). Pokud je dosažen kořen při
hledání prvního tokenu, znamená to, že se programu nepodařilo název na tokeny
rozložit a je vyvolána chybová hláška, v~níž je uveden nejkratší sufix, kterému
se nepodařilo porozumět.

Další zvláštností implementace tohoto modulu je, že nevyužívá sdru\-žo\-vá\-ní tokenů.
Buď by totiž muselo být jejich rozlišení v~sémantické analýze implementováno
pevně, nebo by to vyžadovalo rozšíření jazyka sémantické analýzy o jejich
rozlišování a to kvůli ušetření jednoho (sdružovacího) neterminálu. Přesto je
interface tohoto modulu pro sdružování připraven, protože používá soubor
\soubor{konstanty.txt}, který přiřazuje intervalům tokenů jména.

Tento soubor je navíc využit k~ošetření některých kontextových problémů. Prvním
je vkládání a vypouštění samohlásek ve švu slova, tj. vypouštění koncového ,,a"
nebo ,,o" v~tokenu, popřípadě vkládání ,,o" mezi tokeny. Současná implementace
vypadá tak, že koncové ,,a" je možné vypustit kdykoliv a ,,o" je možné vložit
do švu, pokud se jedná o jméno prvku, charakteristickou skupinu, řeckou číslovku
nekončící samohláskou, kořen triviálního názvu nebo celý triviální název. Druhý
kontextový problém ošetřený v~tomto modulu je existence samostatných hlásek jako
tokenů (např. ,,a" v~,,4a,9a-but[2]enoanthracen" nebo
v~,,dibenzo[{\textit a},{\textit j}]anthracen"). Tyto tokeny nesmějí následovat
po písmenu, protože to by umožňovalo vracet jako tokeny pouze jednotlivé hlásky
(například ,,methan" by bylo rozděleno jako ,,metha$|$n" místo ,,meth$|$an") a
chyba by byla vždy vyvolána až v~syntaktické analýze, namísto v~lexikální.

Práce s~tímto modulem probíhá pomocí několika funkcí. Nejprve je po\-tře\-ba načíst
intervaly tokenů z~\soubor{konstanty.txt}, pro potřeby řešení výše popsaných
kontextových záležitostí na úrovni lexikální analýzy. Toto načtení zajistí procedura
\kod{inicializace\_intervalu}, která v~parametru očekává otevřený soubor.
Následuje načtení seznamu tokenů, přičemž se z~nich postaví lexikální strom. Pro
tyto potřeby slouží procedura \kod{inicializace}. Poté se volá procedura
\kod{parse} se zpracovávaným názvem v~parametru. Ta ho rozdělí na tokeny,
ale pouze v~bufferu. Tyto tokeny jsou pak vraceny postupně včetně svých flagů a
pořadového čísla ve funkci \kod{yylex}. Obsaženy jsou ještě dvě funkce,
\kod{hodnota} a \kod{text}, které převádí číselný kód
terminálu na jeho jméno a naopak. To je využito hlavně v~modulu \modul{Parser}
k~načtení gramatiky.

%%%%%%%%%%%%%%%%%%%%%%%%%%%%%%%%%%%%%%%%%%%%%%%%%%%%%%%%%%%%%%%%%%%%%%%%%%%%%%%%
\section{\modul{Parser}}
\modul{Parser} je modul zajišťující syntaktickou analýzu názvu. Nejprve se
provede inicializace a to tak, že se načte ze souboru \soubor{gramatika.gra} vše
potřebné, tedy gramatická pravidla a jejich sémantický význam (sekvence příkazů).
Při načítání se rovnou převádí hodnoty terminálů na kódy tokenů a stejně tak
jména neterminálů. Těm je navíc potřeba ještě kódy přiřazovat, což se děje
v~pořadí, v~jakém se vyskytují na vstupu. Sémantická pravidla se nijak
nezpracovávají, ponechávají se ve své textové podobě. O jejich interpretaci se
stará až v~případě potřeby modul \modul{Semantika}. Inicializace je zakončena
testem, zda nejsou obsaženy neterminály, pro které neexistuje žádné odvozovací
pravidlo a případně se tento nedostatek ohlásí uživateli.

Práce při hlavním volání vypadá následovně. Program dokola volá
\kod{yylex} dokud dostává tokeny a přitom z~nich tvoří graf typu cesta,
kde tokeny tvoří orientované hrany. Následuje budování derivačního stromu nad
tímto grafem. To probíhá tak, že se postupně prochází pravidla a u každého se
zjišťuje, zda je možné ho aplikovat, tj. zda pravá strana pravidla odpovídá
orientované cestě v~dosavadním grafu. Pokud ano, přidá se hrana spojující
začátek a konec této cesty, přičemž její hodnota odpovídá kódu neterminálu na
levé straně pravidla. Takto se hledají hrany, dokud existuje možnost, že se
nějaká přidá a zároveň se ještě nepodařilo aplikovat pravidlo, které by
vytvořilo hranu počátečního neterminálu přes celou původní cestu.

Zároveň s~tímto grafem se buduje derivační strom a to tak, že vrcholy jsou
tvořeny hranami původního grafu a hrany v~něm vedou z~A do B, pokud byla hrana
odpovídající B použita při vytváření hrany odpovídající A. Zároveň se odstraní
všechny vrcholy (a do nich, případně z~nich, vedoucí hrany), které nejsou
součástí hlavního stromu.

\begin{figure}[h]
	\imgbox{syntax_gramatika}
	\caption{Příklad gramatiky}
\end{figure}

\begin{figure}[h]
	\imgbox{syntax_graf}
	\caption{Vygenerovaný graf pro slovo ABCD}
\end{figure}

\begin{figure}[h]
	\imgbox{syntax_strom}
	\caption{Výsledný derivační strom}
\end{figure}

S~ostatními moduly probíhá komunikace následovně. K~inicializaci slouží
procedura \kod{inicializace}, která načte gramatiku ze souboru, který
očekává otevřený ve svém parametru. Pro překlad kódu symbolu (terminálu nebo
neterminálu) na jeho textovou reprezentaci slouží funkce \kod{kod2str}.
To je využito pouze při chybových výstupech. Komunikaci s~okolím zajišťuje ještě
funkce \kod{yyparse}, která vrací ukazatel na typ \kod{derivacni\_strom::derivat}.
Tato třída je velmi jednoduchá, obsahuje pouze seznam ukazatelů na potomky,
iterátor na pravidlo gramatiky, které je daným uzlem reprezentováno a rozsah
indexů tokenů, jež do daného podstromu spadají -- tedy veškeré informace potřebné
k~dalšímu zpracování.

%%%%%%%%%%%%%%%%%%%%%%%%%%%%%%%%%%%%%%%%%%%%%%%%%%%%%%%%%%%%%%%%%%%%%%%%%%%%%%%%
\section{\modul{Semantika}}
Tvorba modelu molekuly je započata vytvořením jejího grafu. To se děje v~modulu
\modul{Semantika}. Ten nejprve zavolá funkci \kod{yyparse}, která vrátí
de\-ri\-vač\-ní strom zpracovávaného názvu, ten vyhodnotí a strom zruší. Výsledek
vyhodnocení je poté vrácen k~dalšímu zpracování.

Vyhodnocování probíhá tak, že se volá rekurzivně na podstromy a z~na\-vrá\-ce\-ných
dat se staví pole. V~aktuálně zpracovávaném uzlu se pak vezme sekvence příkazů
jemu příslušící (z~gramatiky v~modulu \modul{Parser}) a ta se počne aplikovat na
předem vytvořené pole. Jak postupují mezivýsledky stromem nahoru, nakonec se
dostane výpočet i do kořene, který svůj výsledek vrátí jako výsledek celého
vyhodnocování. Návratová hodnota je struktura, která mimo jiné obsahuje položku
typu \kod{skelet}, ve které je uložena požadovaná molekula (tedy
pouze její struktura, jména atomů, atd.).

Celkově se postupuje tak, že se molekula buduje bez vodíků a až když je vše
ostatní hotovo, připojí se na ,,volné" pozice vodíky. Pokud se stane, že
pravidlo některého uzlu neobsahuje žádné příkazy, pak je na tuto situaci
uživatel upozorněn. Jazyk sémantiky je velice jednoduchý, aby byla jeho
interpretace co nejsnazší. Jeho podrobný popis je v~Příloze~\ref{priloha:jazyk}.

Pro správnou interpretaci je nutné znát některé základní charakteristiky
použitých prvků. Ty jsou načteny ze souboru \soubor{prvky.gra} při inicializaci
(procedura \kod{inicializace}, která očekává v~parametru otevřený
soubor). Analýza se pak spouští pomocí volání funkce \kod{yysem},
která vrací graf molekuly typu \kod{skelet}.

%%%%%%%%%%%%%%%%%%%%%%%%%%%%%%%%%%%%%%%%%%%%%%%%%%%%%%%%%%%%%%%%%%%%%%%%%%%%%%%%
\section{\modul{Model}}
Z~grafu molekuly, který vrátí modul \modul{Semantika} je potřeba vytvořit model.
To znamená určit polohy jednotlivých atomů v~prostoru, což zajišťuje tento
modul. Zmíněnou úlohu je velmi obtížné přesně vyřešit a tak byl zvolen
následující model.

V~molekule dominují dvě síly. Atomy se v~důsledku elektrického náboje odpuzují,
ale jako protiváha je drží pohromadě vzájemné vazby. Stejně tak působí
v~použitém modelu dvě protichůdné síly. Každá dvojice atomů na sebe působí podle
Coulombova zákona, přičemž jejich náboje jsou přímo úměrné protonovému číslu $Z$
(což odpovídá realitě). Vazebné síly jsou pak modelovány jako guma napínaná mezi
atomy. Využity jsou tedy následující dva vzorce. Prvním je Coulombův zákon:
$$F_e = \frac{1}{4\pi\varepsilon}\frac{Q_1 Q_2}{r^2}, \mathrm{kde}$$
\begin{itemize}
	\item $\varepsilon$ je permitivita vakua (konstanta),
	\item $Q_i$ jsou náboje jednotlivých atomů ($Q_i=e_0 Z_i$, kde $e_0$ je
	      náboj elektronu),
	\item $r$ je vzdálenost mezi atomy.
\end{itemize}

Druhý vzorec pak popisuje sílu mezi vázanými atomy následovně:
$$F_p = K \Delta y, \mathrm{kde}$$
\begin{itemize}
	\item $K$ je lineární tuhost (materiálová konstanta),
	\item $\Delta y$ je vzdálenost mezi atomy zmenšená o délku gumy (ta je
	      stanovena experimentálně na hodnotu $4,0$).
\end{itemize}

Dalšího zjednodušení bylo dosaženo nahrazením spojité simulace diskrétní
variantou. Program v~daném okamžiku vždy spočítá výslednici sil působících na
atomy ($F$). Tuto sílu pak nechá konstantně působit po dobu $t_i$ (jedná se tedy
o~rovnoměrně zrychlený/zpomalený pohyb). Takto je zjištěna nová poloha
($P_{i+1}=P_i + v_i t_i + \frac{F_i t_i^2}{2m}$) a rychlost
($v_{i+1}=v_i + \frac{F_i}{m}$) atomů a výpočet se opakuje, tentokrát se však
síly nechají působit po dobu $t_{i+1}$. Posloupnost $t_i$ je navržena tak, aby
výpočet nutně konvergoval (použita byla geometrická posloupnost s~kvocientem
$0,999$ a hodnotou prvního členu $0,01$).

Ačkoliv to není zřejmé, významný vliv na výsledek má i inicializace. Je potřeba,
aby proběhla co nejrychleji, zároveň však musí umožnit rychlý přesun všech atomů
do ,,správné" (tj. v~rámci modelu co nejkvalitnější mož\-né) konfigurace.
Nejjednodušší varianta, jak atomy rozmístit, je umístit je do řady vedle sebe
(například na souřadnice $\left[4i, 0, 0\right]$). To by ale mělo za následek,
že by všechny působící síly měly směr rovnoběžný s~touto přímkou, tudíž by
všechny atomy zůstaly na této přímce. Podobný výsledek by byl dosažen, pokud by
všechny atomy ležely v~jedné rovině. Další zvažovanou variantou bylo rozmístění
podle vzorce:
$$	\left\{\left[x_i, y_i, z_i\right]\right\}_i :=
	\left[ i, i\left(-1\right)^i, i^2\left(-1\right)^i\right],$$
kde $\left[x_i, y_i, z_i\right]$ jsou souřadnice $i$-tého atomu. Při návrhu
tohoto vzorce byl zohledněn fakt, že indexy atomů spojených vazbou se většinou
liší o jedna a že je dobré inicializovat je tak, aby jim v~pohybu k~sobě nic
nebránilo. Tato inicializace však byla dostačující pouze pro molekuly s~méně než
40 atomy. U větších modelů byly již vzdálenosti příliš velké na to, aby se model
během krátké doby po kterou byl simulován stihl ustálit. Hlavním důvodem je
evidentně kvadratická závislost max. souřadnice na počtu atomů. Bylo tedy třeba
navrhnout jinou inicializaci, která by oproti předchozí variantě zohlednila i
potřebu umístit všechny atomy relativně blízko počátku (resp. atomů s~nimiž je
ve vazbě).

Zdá se, že nápad vytvořit počáteční pozici atomů jako rovnoměrné pokrytí sféry
má potenciál řešit všechny výše uvedené požadavky. Přináší však ně\-ko\-lik problémů
nových:
\begin{itemize}
	\item Jaký má mít sféra poloměr?
	\item Jak umístit atomy mezi nimiž je vazba blízko sebe?
	\item Jak rozmístit atomy rovnoměrně?
\end{itemize}

Nejproblematičnější z~těchto otázek je ta poslední -- neexistuje totiž algoritmus
řešící tento problém. Zde jsou však přibližná řešení naprosto do\-sta\-ču\-jí\-cí a
je-li využito to, které atomy rozmístí na sféru rovnoměrně po spirále, je navíc
vyřešena i druhá otázka z~výše uvedeného seznamu.\cite{Saff:koule} Zbývá tak
pouze určit vhodný poloměr použité sféry. Poloměr s~minimální energií lze
odhadnout následovně.

Předpokládejme, že atomy jsou rozmístěny rovnoměně. Tuto situaci lze odhadnout tak,
že tvoří triangulaci, jejíž všechny stěny jsou rovnostranné trojúhelníky stejné
velikosti. Hledáme-li stav s~minimální energií, pak musí mít každá hrana délku
odpovídající optimální délce vazby, kterou známe. Nyní jsme již schopni dopočítat
požadovanou plochu každé stěny, požadovaný povrch sféry a tedy i její poloměr.
Při použití tohoto poloměru nicméně dochází k~situacím, kdy systém od začátku nemá
dostatek energie pro dobré formování. Z~toho důvodu byla celá sféra o proti výše
uvedenému odhadu několikanásobně zvětšena.

Pro další zjednodušení modelu (resp. počítání v~něm) se neuvažují sku\-teč\-né
hodnoty konstant (náboj, hmotnost, permitivita vakua, \dots), ale jsou nahrazeny
hodnotami vyššími. Tím, že jsou takto upraveny všechny konstanty ve správném
poměru, je zachováno chování modelu, ale výpočty jsou přesunuty z~řádu
$10^{-27}$ do řádu jednotek, kde je dosahováno vyšší přesnosti. Výpočty tedy
probíhají podle následujících vzorců:
\begin{eqnarray*}
	F_e&=&\frac{Z_1 Z_2}{r^2} \\
	F_p&=&r\left(1-\frac{4,0}{r}\right)\;\left(=\Delta y\right)
\end{eqnarray*}

\begin{figure}[h]
	\imgbox{fyz_model}
	\caption{Vzájemná interakce mezi dvěma atomy spojenými vazbou podle modelu}
\end{figure}

Interface tohoto modulu není příliš rozsáhlý. Hlavní je metoda
\kod{yymodel}, která vypíše pomocí následujícího modulu \modul{Grafika}
výsledek své práce -- vytvořený model molekuly. Ten vznikne tak, že se načte
graf molekuly z~parametru do vlastní třídy \kod{skelet} a zavolá se její
metoda \kod{optimalize}, která pracuje výše popsaným způsobem.

%%%%%%%%%%%%%%%%%%%%%%%%%%%%%%%%%%%%%%%%%%%%%%%%%%%%%%%%%%%%%%%%%%%%%%%%%%%%%%%%
\section{\modul{Grafika}}
Tento modul zajišťuje výstup. Slouží k~tomu jeho funkce \kod{tisk},
jejíž parametr je model a vrácená hodnota odpovídá jeho VRML zápisu.

V~hlavičce (která je v~souboru \soubor{sablona.wrl} a načítá se do paměti funkcí
\kod{inicializace}) jsou předdefinovány pohledy (VIEWPOINT) a prototypy
(PROTO) pro Atom (\kod{Atom}) a jednotlivé typy vazeb (\kod{Vazba1},
\kod{Vazba2} a \kod{Vazba3}) (viz {\appendixname}~\ref{priloha:vrml}). Následují
definice atomů skutečně použitých a to jako specifikace obecného \kod{Atom} (je
určena barva, značka a~velikost). Poté již je vypsána konkrétní molekula. Každý
atom je navíc instanciován s~parametrem obsahujícím seznam všech lokantů, které
mu během stavby molekuly příslušely, s~tím, že ty, které již nejsou platné, jsou
prefixovány vykřičníkem. Pro snazší zacházení nejsou struktury vnořeny (stejně
by bylo obtížné určit, kterému ze dvou vázaných atomů vazba náleží).

Pro určení transformace vazby do požadované polohy, bylo třeba stanovit osu a
úhel otočení ze své počáteční (s~osou y rovnoběžné) pozice. Pro snazší
manipulaci je vazba (válec/válce) nejprve posunuta dolní podstavou do po\-čát\-ku
(v~generované pozici leží počátek lokálního souřadného systému uprostřed vazby).
Nakonec je posunut do prvního z~vázaných atomů (zde se projeví výhoda prvního
posunu).

\begin{figure}[h]
	\imgbox{transformace_vazby}
	\caption{Výpočet parametrů pro otočení vazby do správného směru}
\end{figure}

Výpočet parametrů otočení (tedy osy a úhlu) je relativně jednoduchý a vychází
z~následujících předpokladů:

Značení: $A$ -- původní směr vazby,
		 $B$ -- transformovaný směr vazby,
		 $o$ -- směr osy rotace,
		 $\phi$ -- velikost rotace,
		 $L$ -- délka vazby $\left(=\left\|A\right\|\right)$.

\zlom
\begin{eqnarray*}
o&=&B \times A \\
\cos (\phi) &=& \frac{A\cdot B}{\|A\|\cdot \|B\|} \\
A \times B &=& \left[ a_2 b_3 - a_3 b_2, a_3 b_1 - a_1 b_3, a_1 b_2 - a_2 b_1 \right] \\
A &=& \left[ 0, L, 0\right]
\end{eqnarray*}

Po dosazení posledního vztahu do předchozích a následném vyjádření neznámých
jsou získány kýžené parametry transformace:
\begin{itemize}
	\item $o=\left(-b_3, 0, b_1\right)$ (u osy je nutné zajistit pouze směr
		  vektoru, nikoliv jeho velikost, proto je možno celý výraz zjednodušit
		  o jistě nenulovou hodnotu $L$),
	\item $\phi = \arccos \left(\frac{b_2}{\|B\|}\right)$.
\end{itemize}

Násobnost vazby je vyznačena počtem válců. Aby byl výsledný vzhled přirozený,
bylo potřeba dosáhnout opticky shodné síly vazeb, zároveň však bylo potřeba
učinit je snadno rozlišitelnými. To se snad podařilo tak, že průřezu každé vazby
lze opsat jednotkovou kružnici a střed podstavy každého válce leží na hranici
válců ostatních (viz {\figurename}~\ref{pic:vazba}).

\begin{figure}[h]
	\imgbox{vazba}
	\caption{Odvození tvaru násobných vazeb v~modelu}\label{pic:vazba}
\end{figure}

