\chapter{Úvod}
Chemie je dnes jedním z~nejrychleji se rozvíjejících a nejvíce využívaných
vědních oborů. Uplatnění nachází takřka všude, například v~potravinářství
(potravinářská chemie), v~každé výrobní hale (chemie materiálů), při výrobě
léčiv (farmakologická chemie), umělých cév, hnojiv, polovodičů nebo vý\-buš\-nin.

Chemie se dělí na anorganickou a organickou. Původně toto rozdělení bylo velmi
intuitivní, protože vzájemná přeměna mezi sloučeninami organickými (ze živé
přírody) a anorganickými (z~neživé) byla neznámá, ale od roku 1828 jsou tyto
přeměny známy a hranice tedy musí vést jinudy. V~současnosti je definována
organická chemie jako \strong{chemie sloučenin uhlíku}.

Velmi dlouhou dobu byla takřka veškerá pozornost soustředěna na chemii
anorganickou. Poté, co nové znalosti vědecké činnosti v~organické chemii začaly
nacházet své uplatnění v~komerční průmyslové praxi, zájem se pře\-su\-nul
k~organické chemii. Dnes pozorujeme přesně opačný trend -- neuvěřitelný rozvoj
tohoto odvětví a~jeho uplatňování v~oborech, kde by to dříve nikdo nečekal
(například izolační materiály ve stavitelství).

S~tím souvisí potřeba rychlejší, ale zároveň přesné komunikace. Pojmenovat
13\,milionů známých sloučenin tak, aby podle jména byla snadno rozpoznatelná
vnitřní struktura a zároveň si při tom nechat prostor pro jména nově
vznikajících sloučenin, není zrovna snadné. Pro tyto účely existuje systematické
názvosloví. Vedle systematických názvů existují ještě názvy tradiční,
semisystematické nebo triviální, které by sice bylo možné nahradit, ale za cenu
ztráty srozumitelnosti (benzen by se označoval jako cyklohexa-1,\,3,\,5-trien).

Jako systematické názvosloví se označuje pojmenování sloučenin podle pravidel
vytvořených a publikovaných IUPAC (Mezinárodní unie pro čistou a~aplikovanou
chemii). Český překlad této tzv. ,,Modré knihy" (z~anglického originálu) tato
pravidla zároveň přizpůsobuje potřebám českého jazyka.

Poslední aktualizace systematického názvosloví proběhla v~roce 1993
(čes\-ky 2000)\cite{IUPAC:bluebook,Kahovec:modrakniha}. Vedena byla snahou více
zesystematizovat současné názvosloví, odstranit potenciální místa nedorozumění
a hlavně sílící potřebou nejen jednoznačnosti, ale také jedinečnosti názvů.
Současný stav, kdy pro většinu sloučenin existuje více variant pojmenování, dělá
velké problémy například při sestavování rejstříků (vznik křížových odkazů a
vícenásobných hesel)\cite{Schejbalova:organa}.

Český překlad pak využil příležitosti ,,nutné" aktualizace a začlenil navíc do
nových pravidel změny zápisu tam, kde šlo o neopodstatněnou odchylku od anglické
předlohy. Negativními důsledky jsou neznalost a nedodržování -- zvláště když se
jedná pouze o kodex a nikoliv o normu. Jen málo lidí dnes ovládá správné české
organické názvosloví a s~chybami je možné se setkat nejen u laické veřejnosti či
publicistů, ale dokonce i u lidí přednášejících chemii na vysokých školách nebo
na příbalových letácích léků.

Motivací pro vznik této práce byla výše zmíněná problematika. Cílem bylo
vytvořit prostředí, ve kterém by si mohl uživatel zobrazit prostorový model
organické sloučeniny podle jejího českého názvu. Přičemž práce neměla ambice
poskytnout realistický model ani pokrýt celé organické názvosloví (protože obě
tyto problematiky jsou značně rozsáhlé), ale byla vedena tak, aby mohla být do
takového rozsahu dodatečně rozšířena.
