\chapter{\label{priloha:skripty}Skripty}

\section{START.bat}
\begin{kodblok}
\klic{@echo} off
\klic{cd} Complet \bs
\klic{@echo} on
\klic{start} server 80
\klic{start} firefox "http://127.0.0.1/chenom"
\end{kodblok}

\section{START.sh}
\begin{kodblok}
\klic{#!/bin/sh}
\klic{cd} Complet
./server 8080&
firefox "http://127.0.0.1:8080/chenom"
\end{kodblok}

\section{run.bat}
\begin{kodblok}
\koment{rem @echo off}
\klic{set} jmeno=%1
\klic{echo} %jmeno% >input
\klic{set} vystup="outs{\bs}%jmeno:~1,-1%.wrl"

\klic{if} EXIST %vystup% \klic{goto} cached

\koment{rem ------------------}
\koment{rem necachovat modely:}
\koment{rem del /F /Q "outs{\bs}*.wrl"}
\koment{rem ------------------}

chenom.exe DATA{\bs}cp1250{\bs}settings.ini -e >%vystup% 2>error.log < input
\klic{if} ERRORLEVEL 1 (
    \klic{del} input

\koment{rem ------------------------}
\koment{rem podrobny chybovy vystup:}
    \klic{echo} ===================================== >> error.log
    \klic{type} %vystup% >> error.log
\koment{rem ------------------------}

    \klic{del} %vystup%
    \koment{rem pause}
    \klic{@echo} on
    \klic{exit} 3
)
\underline{:cached}
    \klic{del} input
\klic{@echo} on
\end{kodblok}

\section{run.sh}
\begin{kodblok}
\klic{#!/bin/sh}
vstup=input
vystup="outs/$@.wrl"

\klic{echo} $@ >$vstup

\koment{# ------------------}
\koment{# necachovat modely:}
\klic{rm} -f outs/*.wrl
\koment{# ------------------}

\klic{if} [ ! -e $vystup ]; \klic{then}
  \klic{if} ! ( \bs
  ./chenom DATA/utf8/settings.ini -e <$vstup >$vystup 2>error.log \bs
  ); \klic{then}
    \koment{# ------------------------}
    \koment{# podrobny chybovy vystup:}
    \klic{echo} ======================== >> error.log
    \klic{cat} $vystup >> error.log
    \koment{# ------------------------}
    \klic{rm} -f $vstup
    \klic{rm} -f $vystup
    \klic{exit} 3
  \klic{fi}
\klic{fi}
\klic{rm} -f $vstup
\end{kodblok}

\section{run-console.bat}
\begin{kodblok}
\klic{@echo} off
\klic{set} jmeno=%1
\klic{echo} %jmeno% >input
\klic{set} vystup="outs{\bs}%jmeno:~1,-1%.wrl"
\klic{if} EXIST %vystup% goto cached

\koment{rem ------------------}
\koment{rem necachovat modely:}
\koment{rem del /F /Q "outs{\bs}*.wrl"}
\koment{rem ------------------}

chenom.exe DATA{\bs}cp1250{\bs}settings.ini >%vystup% < input
\klic{set} chyba=%ERRORLEVEL%
\klic{if} ERRORLEVEL 1 (
  \klic{del} input

  \koment{rem ------------------------}
  \koment{rem podrobny chybovy vystup:}
    \klic{echo} =========================== >> error.log
    \klic{type} %vystup% >> error.log
  \koment{rem ------------------------}

  \klic{del} %vystup%
  \klic{pause}
  \klic{@echo} on
  \klic{exit} /B %chyba%
)
\underline{:cached}\hfill \\

\klic{del} input
\koment{rem --------------------------}
\koment{rem zde napiste prikaz, ktery otevre model ve vasem prohlizeci}
\koment{rem cesta k souboru je %vystup%}
  \koment{rem explorer %vystup%}
  %vystup%
\koment{rem --------------------------}
\koment{rem necachovat (druha varianta):}
    \koment{rem del %vystup%}
\koment{rem --------------------------}
\klic{@echo} on
\end{kodblok}

\section{run-console.sh}
\begin{kodblok}
\klic{#!/bin/sh}
\klic{echo} $@ >input
vstup=input
vystup="outs/$@.wrl"

\koment{# ------------------}
\koment{# necachovat modely:}
\klic{rm} -f outs/*.wrl
\koment{# ------------------}

\klic{if} [ ! -e $vystup ]; \klic{then}
  \klic{if} ! ( {\bs}
    ./chenom DATA/utf8/settings.ini <$vstup >$vystup 2>error.log {\bs}
  ); \klic{then}
    \koment{# ------------------------}
    \koment{# podrobny chybovy vystup:}
    \klic{echo} ======================== >> error.log
    \klic{cat} $vystup >> error.log
    \koment{# ------------------------}
    \klic{rm} -f $vstup
    \klic{rm} -f $vystup
    \klic{exit} 3
  \klic{fi}
\klic{fi}
\klic{rm} -f $vstup

\koment{# zde napiste prikaz, ktery otevre model ve vasem prohlizeci}
\koment{# cesta k souboru je $vystup}
\end{kodblok}
