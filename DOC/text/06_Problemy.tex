\chapter{Problémy implementace}
Je zřejmé, že není možné implementovat obecnou a úplnou interpretaci organického
názvosloví. Při práci na programu CheNom však byla vyvinuta snaha o~možnost
pozdějšího rozšíření o všechny (více či méně dobrovolně) opomenuté aspekty a to
nejlépe bez nutnosti měnit kód programu (a tedy rekompilovat), ale pouhým
zásahem do datových souborů. Z~tohoto přístupu vyplynuly následující problémy.

\section{Přidávání triviálních názvů}
Protože není možné do programu zanést všechny triviální názvy, je nutné umožnit
jejich co nejsnazší přidávání. Tento požadavek přinesl několik komplikací.
Modifikace zdrojového kódu programu rozhodně není snadné přidání. Je tedy nutné
toto provádět ve vnějším (datovém) souboru. Nejedná se však pouze o triviální
názvy, uživatel může potřebovat přidat třeba funkční skupinu. Pak je tedy
potřeba umístit do vnějšího souboru celou gramatiku. Tím odpadá možnost využití
pomocných nástrojů jako \todo{flex} či \todo{bison}. Navíc je nutné v~tomto souboru
s~gramatikou popsat sémantiku jednotlivých pravidel. Pro tento jazyk je nutné
vytvořit interpret, a proto je vhodné vytvořit malý, snadno analyzovatelný
jazyk s~výrazovými prostředky přesně vytvořenými pro potřeby popisu sémantiky
názvotvorných operací.

\section{Třída jazyka organické chemie}
Existence úprav jako nahrazení koncovky ,,-ová kyselina" koncovkou ,,-át" nebo
vkládání infixů do názvů (např. záměnný infix ,,-imido-") dělá jazyk organické
chemie výrazně kontextovým. Tato skutečnost velice komplikuje jak lexikální, tak
syntaktickou analýzu. Situaci neprospívá ani nejednoznačnost pravidel. Například
pravidlo R-0.1.7.3 popisuje vkládání ,,o" následovně: \citace{Pro lepší
výslovnost a libozvučnost se někdy mezi souhlásky předpony vkládá samohláska
\todo{-o-}}~\cite{Kahovec:modrakniha}. Také obraty jako \citace{zvažuje komise} nebo
\citace{bude popsáno v~budoucí publikaci} nejsou
výjimečné~\cite{Kahovec:modrakniha}.

Některé z~těchto problémů se podařilo odstranit v~lexikální analýze. Jako součást
lokantů v~názvosloví kondenzovaných polycyklů se používají písmena latinky. Při
bezkontextovém zpracování je tak možné rozdělit každý název tak, že jednomu
tokenu odpovídá jeden znak. Částečně lze tomuto zabránit tím, že se místo
obvyklé varianty použití prvního (nejkratšího) nalezeného tokenu použije
nejdelší. I tak by ale například analýza názvu ,,methan" vrátila tokeny ,,metha"
a ,,n". Kontextové pravidlo umožňující vrátit token odpovídající jednomu písmenu
latinky pouze v~případě, že předchozí token nekončí písmenem, řeší i tento
nedostatek. Dalším kontextovým pravidlem uplatněným v~lexikální analýze je již
zmíněné vypouštění a vkládání samohlásek.

Problémy s~kontextovostí infixových názvotvorných operací se nepodařilo
odstranit, a proto bylo od jejich implementace upuštěno.

Syntaktická analýza byla zvolena pro co nejširší třídu jazyků a tedy
bezkontextová nedeterministická na způsob \todo{Q-systémů}.

\section{Názvy činící komplikace}
Sestavení gramatiky názvosloví organické chemie tak, aby pokud možno
ne\-ob\-sa\-ho\-va\-la žádné konflikty (a to ani po jejím rozšiřování) je velmi náročné.
Například \chem{HI} lze označit jako jodan, \chem{I-} pak jako jodyl. Jodid se
zdá být obdobou jodylu a tedy \chem{I^-}. Ve skutečnosti je však jodid název
charakteristické skupiny \chem{I-} a pro \chem{I^-} je přípustné pouze označení
jodanid.

Větším problémem je spojení ,,karboxylová kyselina". Jak je totiž me\-tha\-n\-al a
methylaldehyd dvojí označení téhož, tak v~důsledku shody funkčního skupinového
názvu a substituční přípony charakteristické skupiny před\-sta\-vu\-jí názvy
,,methankarboxylová kyselina" a ,,methylkarboxylová kyselina" tutéž sloučeninu.
Ovšem byla-li by chápána druhá varianta ve smyslu první, pak by byla na
připojovaném uhlíku vytvořena nevyužitá volná vazba (v~ta\-ko\-vém případě by bylo
správnější označení ,,1-oylmethylkarboxylová kyselina").

Mnoho komplikací činí také několik skupin názvů, které se mohou zdát při
nejmenším zavádějící. To samozřejmě komplikuje i jejich automatické zpracování.
Patří mezi ně například:
\begin{itemize}
	\itemB{ethylen} (\chem{-CH_2-CH_2-}) -- není možné interpretovat jako
					  ethenyl (též vinyl, \chem{CH_2=CH-}).
	\itemB{rhodanid} -- jedná se o zastaralé označení pro thiokyanát a souvislost
					   s~rhodiem je jen zdánlivá.
	\itemB{rhodanin} -- synonymum pro 2-thioxothiazolidin-4-on, tedy opět
					   žádné rhodium.
	\itemB{per} -- infix ,,-peroxo-" znamená náhradu \chem{-O-} za \chem{-O-O-},
				 ale ,,per" samo o sobě má význam zcela, takže předpona
				 ,,perhydro-" vyjadřuje proces nasycení všech vazeb. Druhý
				 z~významů je však podle do\-po\-ru\-če\-ní 1993 již zastaralý a neměl
				 by se tedy užívat.
	\itemB{S} -- označuje stereodeskriptor chirálních sloučenin, ale také se
				jedná o značku síry.
	\itemB{fluoren} -- triviální název tricyklické sloučeniny, který může budit
					  dojem fluoru a dvojné vazby.
	\itemB{propadien} -- tento název sice není zavádějící, přesto je problematický
					 -- ,,dien" je totiž ekvivalentní název pro ethen. ,,propa-"
					 je pak možné chápat jako předponu.
\end{itemize}

\section{Konstituce, konfigurace, konformace}
Implementována musí být konstituce, protože jeden sumární vzorec může označovat
velké množství látek naprosto odlišných vlastností. Aby mohla být implementována
konfigurace, je nutné správně modelovat (nemá smysl řešit, zda má být atom
umístěn nad nebo pod rovinou cyklu, pokud je při modelování umístěn do této
roviny). Stejně jako konfigurace ani konformace nebyly implementovány. Hlavní
důvod je ten, že v~drtivé většině případů nemá názvosloví nástroje je od sebe
odlišit (například vaničková a židličková konformace cyklohexanu).

