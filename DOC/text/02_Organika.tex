\chapter{Problematika organické chemie}
\section{Názvosloví}
Organická chemie pracuje oproti anorganické s~mnohem větším množstvím sloučenin.
Ty mohou navíc tvořit výrazně složitější struktury. Z~toho vyplývá, že
i názvosloví je komplikovanější.

Názvosloví anorganické chemie je (speciálně v~češtině) výrazně stručnější.
Existuje jen několik málo druhů sloučenin (kyseliny, zásady, soli, \dots), navíc
tyto sloučeniny z~pravidla obsahují jen malý počet atomů, takže není nutné blíže
popisovat vnitřní strukturu (je totiž i tak jednoznačná). O tuto pře\-hled\-nost se
zasloužil Dr. Emil Votoček, který využil toho, že je čeština flexní typ
syntetického jazyka a užil koncovek k~zanesení informace o valencích atomů do
názvu.

Názvy v~organické chemii se dělí na triviální, semisystematické/se\-mi\-tri\-vi\-ál\-ní
a systematické. Triviální název je název, jehož žádná část nepochází ze
systematického názvosloví (např. močovina). Semisystematický název je název,
v~němž alespoň jedna část je vytvořena v~systematickém smyslu (např. glycerol [ol],
butan [an]). Systematický název na rozdíl od triviálního vyjadřuje systematické
zařazení popisovaného jevu, v~tomto případě se řídí chemickou strukturou látky.
S~pojmem struktura souvisí další pojmy, a to konstituce (vzájemné spojení atomů
vazbami a typy vazeb), konfigurace (uspořádání atomů v~prostoru, které není
možné měnit volnou rotací kolem jednoduché vazby) a konformace (uspořádání atomů
v~prostoru, které je možné měnit volnou rotací kolem jednoduché vazby), jimž je
pojem struktura nadřazen\cite{Kolar:organa}.

Systematické názvy se tvoří pomocí několika názvoslovných operací. Vše\-chny
pracují na stejném principu: pomocí předpony (popř. přípony, vý\-ji\-me\-č\-ně vsuvky)
aplikované na původní název, vzniká název sloučeniny po modifikaci příslušnou
operací.

Názvoslovné operace se dělí na
\begin{itemize}
	\item substituční -- nahrazení jednoho nebo více atomů vodíku jiným atomem
						 nebo skupinou atomů (upřednostňovaná operace) \\
						(např. \chem{CH_4} [methan] $\Rightarrow$ \chem{CH_3Cl} [chlormethan]),
	\item záměnné -- výměna jedné skupiny atomů nebo jednoho atomu jiného než
					 vodíku za jinou skupinu atomů, resp. za jiný atom. \\
					 (např. \chem{CH_3-CH_3} [ethan] $\Rightarrow$ \chem{SiH_3-CH_3} [silaethan]),
	\item aditivní -- formální skládání názvu z~částí beze ztráty atomů nebo
					  skupin z~kterékoli části \\
					  (např. \img{naftalen} [naftalen] + 4H $\Rightarrow$ \img{1,2,3,4-tetrahydronaftalen} [1,\,2,\,3,\,4-tetrahydronaftalen]),
	\item konjunktivní -- formální spojení názvů jednotlivých složek; odtržení
						  stejného počtu atomů vodíku z~každé složky v~každém
						  místě spojení \\
						  (např. \img{benzen} [benzen] + \chem{2COOH} [mravenčí kyselina] $\Rightarrow$ \img{benzen-1,4-dimravenci_kyselina} [benzen-1,4-dimravenčí kyselina]),
	\item subtraktivní -- odstranění atomu(ů), iontu nebo skupiny, jež jsou
						  implicitně zahrnuty v~názvu výchozí sloučeniny \\
						  (např. \chem{CH_4} [methan] $\Rightarrow$ \chem{CH_3-} [methyl]) a
	\item násobicí -- svazy obsahující dvoj- nebo vícevazné substituenty \\
					 (např. \chem{N(CHCl-COOH)_3} [$2,2',2''$-trichlornitrilotrioctová kyselina]).
\end{itemize}

\section{Grafická reprezentace molekul\cite{Mysicka:vzorce}}
Při snaze interpretovat chemický název vzniká otázka, co má být výsledkem.
Existuje několik v~praxi používaných možností. Obecně je možné rozdělit je na
rovinné chemické vzorce (obvyklé v~literatuře) a prostorové modely. Navíc
existuje ještě jakýsi mezistupeň -- chemické vzorce, které pomocí projekce
znázorňují model.

\begin{figure}
	\imgbox{vzorce}
	\caption{Vývoj symboliky sloučenin (jako příklad je uveden ethanol)
				\cite{Dalton:newsystem}
			}
\end{figure}
\begin{figure}
	\imgbox{benzeny}
	\caption{\label{pic:benzeny}Návrhy vzorce pro molekulu benzenu.}
\end{figure}

Chemické vzorce jsou dvourozměrné znakové modely sloučenin. Umožňují vyjádřit
kvalitativní složení, jak jsou jednotlivé atomy v~molekule vzájemně vázány a
jaké je jejich prostorové uspořádání. Strukturu lze však zachytit i pomocí
jiných symbolů, jako například topologickou maticí orientovaného grafu nebo
orbitálního diagramu.\zlom

Obecně lze vzorce rozdělit podle míry charakterizace složení a struktury na
\begin{itemize}
	\item stechiometrické (empirické) vzorce -- vyjadřují pouze poměr, v~jakém
		  jsou atomy jednotlivých prvků v~molekule zastoupeny (1 se neuvádí) \\
		  \chemB{CH_2O} -- methanal, ethanová kyselina, glukosa, \dots
	\item molekulové (sumární, souhrnné) vzorce -- oproti stechiometrickým
		  vzor\-cům nezachycují pouze poměry, ale skutečné počty atomů \\
		  \chemB{C_6H_4ClNO_2} -- 2-chlornitrobenzen, 3-chlornitrobenzen, \dots
	\item funkční (racionální) vzorce -- vyjadřují jednotlivé charakteristické
		  (dříve funkční) skupiny \\
		  \chemB{\left(CH_3\right)_2CO} -- propan-2-on
	\item strukturní vzorce -- popisují strukturu, tj. vzájemné vazby a jejich
		  uspořádání; podle toho, zda popisují konstituci, konfiguraci nebo i
		  konformaci se rozlišují
		  \begin{itemize}
				\item Konstituční vzorce \\
					  \img{benzen} -- benzen \\
					  \chemB{CH_3-CH_3} -- ethan \\
					  \img{octova_kyselina} -- kyselina octová (elektronový vzorec) \zlom
				\item Konfigurační vzorce
					  \begin{itemize}
							\itemB{Bočná projekce} -- směr dopředu je vyjádřen
									klínem, směr dozadu přerušovanou čárou \\
									\img{chlormethan} -- chlormethan
							\itemB{Fischerova projekce} -- vazba směřující pod
									(nad) rovinu nákresny je vyjádřena
									vertikálou (horizontálou) \\
									\img{L-glyceraldehyd} -- L-glyceraldehyd
							\itemB{Haworthova projekce} -- síla vazby (ve smyslu
									šířky) zná\-zor\-ňu\-je vzdálenost od pozorovatele \\
									\img{D-glukopyranosa} -- D-glukopyranosa
					  \end{itemize}
				\item Konformační vzorce
					  \begin{itemize}
							\itemB{Perspektivní projekce} \\
									\img{1,2-dichlorethan__perspektivne}
										-- 1,2-dichlorethan
							\itemB{Bočná projekce} \\
									\img{1,2-dichlorethan__bocne}
										-- 1,2-dichlorethan
							\itemB{Newmanova projekce} -- projekce ve směru vazby
									spojující uhlíkové atomy \\
									\img{1,2-dichlorethan__newman}
										-- 1,2-dichlorethan
					  \end{itemize}
		  \end{itemize}
	\item Jiné formy vyjádření struktury
		  \begin{itemize}
				\item Distanční matice \\
					  \img{distancni_matice} -- mravenčí kyselina
				\item Incidenční matice \\
					  \img{incidencni_matice} -- mravenčí kyselina
				\item Grafy -- vrcholy grafu jsou atomy a hrany jsou vazby mezi
					  nimi, volné elektronové páry jsou vyjádřeny smyčkami \\
					  \img{topologie} -- mravenčí kyselina
		  \end{itemize}
\end{itemize}

Modely molekul je možné rozdělit na materiální a počítačové. Stejně jako u
vzorců záleží při výběru typu modelu na účelu jeho použití.

Základními typy materiálních modelů jsou:
\begin{itemize}
	\item kuličkové modely -- atomy jsou reprezentovány kuličkami různé barvy
		  (podle prvku), vazby pak znázorňují tyčky příslušné délky mezi nimi, \\
		  \imgB{model_kulickovy} -- ethan
	\item kalotové modely -- vznikají spojováním čepiček (fr. \strong{calotte}),
		  poloměry odpovídají van der Waalsovým efektivním poloměrům, \\
		  \imgB{model_kalotovy} -- ethan
	\item trubičkové modely -- jedná se o soustavu trubiček nesoucích význam
		  vazby, kde oblasti příslušící jednotlivým atomům jsou vyznačeny
		  zbarvením, \\
		  \imgB{model_trubickovy} --butan
	\item Dreidingovy modely -- atomy jsou představovány kroužky popř. ku\-lič\-ka\-mi
		  s~výstupky, vazby pak trubičkami na ně nasazenými. \\
		  \imgB{model_dreidinguv}
\end{itemize}

Počítačové modely se buď snaží napodobit modely materiální (pak se tedy opět
jedná o modely kuličkové, kalotové, \dots) nebo se snaží zobrazení kombinovat
(např. poloprůhledný kalotový model, uvnitř kterého je zobrazen model
trubičkový). Někdy jde ještě dál za ,,možnosti" materiálního modelu (např.
zobrazení mapy elektrostatického potenciálu, hustoty elektronů na vazbách, atd.).
I tyto modely je možné dále kombinovat mezi sebou.

\begin{figure}[h]
	\imgboxB{model_PC}
	\caption{Přehled některých druhů počítačových modelů benzenu.}
\end{figure}

Ať už je k~zachycení struktury molekuly použita jakákoliv metoda, je vždy
problém u sloučenin se sdílenými elektrony. Jedná se totiž v~podstatě o jednu
vazbu mezi více než dvěma atomy. Pro benzenová jádra existuje u konstitučních
vzorců speciální symbol -- kroužek (viz \picref{pic:benzeny}), ale například ozon
již žádnou takovou značku nemá. Z~nouze se pak v~takových případech používá
některý z~tzv.~,,rezonančních hybridů", což je ovšem trochu zavádějící (výsledná
molekula je totiž kombinací všech těchto hybridů). Tento problém vychází
z~nedostatků Lewisovy teorie lokalizovaných vazeb, která je základem všech výše
zmíněných chemických značení. Představou rezonance lze odstranit jen některé
z~problémů Lewisovy teorie, pro další lze použít představu hypervalence nebo
elektronového deficitu, přesto však u některých molekul nastávají potíže
\cite{Micka:anorgana}.
