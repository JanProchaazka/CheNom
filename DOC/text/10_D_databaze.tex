\chapter{\label{priloha:databaze}Databázové soubory}
\section{settings.ini}
\begin{kodblok}
[Soubory]
Konstanty = konstanty.txt
Terminaly = terminaly.gra
Gramatika = gramatika.gra
Prvky = prvky.gra
Vystup = sablona.wrl
\end{kodblok}

Jméno souboru nesmí obsahovat mezeru, pokud je použita relativní cesta, nesmí
být z~bezpečnostních důvodů použit odkaz do nadřazené složky. Cesta je vztažena
ke~složce, v~níž se nachází tento soubor. Tedy pokud je absolutní cesta k~tomuto
souboru \soubor{C:{\bs}chenom{\bs}DATA{\bs}settings.ini}, ukazuje cesta
\soubor{v2{\bs}prvky} na soubor \soubor{C:{\bs}chenom{\bs}DATA{\bs}v2{\bs}prvky}.

\section{konstanty.txt}
\begin{kodblok}
0000-0006 - Separator \koment{(.,,,:,;,-,->)}
0007-0007 - Carka \koment{(')}
0008-0019 <
0020-0029 - Zavorka \koment{((,),[,],\{,\})}
0030-0039 - Operace \koment{(de,nor,anhydro,cyklo,seko,abeo,retro)}
0040-0049 <
0050-0059 - Formatovani \koment{(<i>,<sub>,...)}
0060-0099 <
0100-0199 - CharakteristickaSkupina \koment{(yl,an,en,yn,...)}
0200-0299 - Stereodeskriptory \koment{(cis,trans,meso,...)}
0300-0499 <
0500-0529 - AnorgPripona \koment{(ný,natý,...)}
0530-0999 <
1000-1009 - Cislice \koment{(0123456789)}
1010-1099 <
1100-1199 - Alfabeta \koment{(\(\mathrm{\alpha\beta}\)...)}
1200-1229 - Latinka \koment{(abcd...)}
1230-1249 - Samohlaska \koment{(aáeéěiíoóuúůyý)}
1250-1399 <
1400-1433 - ReckePocty \koment{(#lokantů--přidat_násobnost_lokantů,...)}
1434-1499 - LatinskePocty \koment{(počty_opakování_struktur,...)}
1500-1999 <
2000-2199 - ZnackaPrvku \koment{(H,He,...)}
2200-2999 - JmenoPrvku \koment{(oxa,sila,...)}
3000-3999 - TypSlouceniny \koment{(aren,oxokyselina,keton,...)}
4000-4999 - TrivKoren \koment{(bora,oxida,joda,metha,...)}
5000-9999 - TrivNazvy \koment{(acetylen,ethan,...)}
\end{kodblok}

\section{terminaly.gra}
\begin{kodblok}
========= 0000
'-'
';'
':'
','
'.'
'->'
'''
--------- 0020
'('
')'
'['
']'
'{'
'}'
--------- 0030
'de'
'nor'
'anhydro'
'cyklo'
'seko'
'abeo'
'retro'
'per'
--------- 0050
'&nbsp;'
'<sup>'
'</sup>'
'<sub>'
'</sub>'
'<i>'
'</i>'
========= 0100
'yl'
'yliden'
'ylidyn'
'ium'
'ylium'
'id'
'it'
'át'
'an'
'en'
'yn'
'al'
'ol'
'on'
'lat'
'at'
'oyl'
'imin'
'amoyl'
'oát'
'os'
  .
  .
  .
\end{kodblok}

\section{gramatika.gra}
\imgrot{gramatika_gramatiky}

\section{prvky.gra}
\begin{kodblok}
--Zn    Z  arita  hybridizace    poloměr  hmotnost  barva
----------------------------------------------------------
>>H     1    1          s1          1       1.E-3  #0000ff
--He    2   -1          -           1       1.E-3  #ffffff
--Li    3   -1          -           1       1.E-3  #ffffff
--Be    4   -1          -           1       1.E-3  #ffffff
--B     5    3          -           1       1.E-3  #ffffff
>>C     6    4          s1p3        1       1.E-2  #ff0000
>>N     7    3          s1p2        1       1.E-3  #00ff00
>>N     7    5          s1p3d1      1       1.E-3  #00ff00
>>O     8    2          s1p1        1       1.E-3  #ffffff
>>F     9    1          s1          1       1.E-3  #00ffff
--Ne   10   -1          -           1       1.E-3  #ffffff
  .     .    .          .           .         .       .
  .     .    .          .           .         .       .
  .     .    .          .           .         .       .
\end{kodblok}

\section{\label{priloha:vrml}sablona.wrl}
\begin{kodblok}
\klic{#VRML V2.0 utf8}
\klic{NavigationInfo} \{ type "EXAMINE" \}
\koment{############################## pohledy ##############################}
\klic{Viewpoint} \{
  position 0 0 50
  description "above"
\}
\komentH{# \dots}

\koment{############################ atom - vzor ############################}
\klic{PROTO} Atom[
    field SFColor   dCol 1.0 1.0 1.0  \koment{# barva atomu}
    field SFFloat   fRad 1.           \koment{# polomer atomu}
    field MFString  sSig []           \koment{# znacka prvku}
    field MFString  sLoc []           \koment{# seznam lokantu atomu}
    ]\{
  Transform\{
    children[
      \klic{DEF} java Script\{
        eventIn   SFBool   popisky
        eventOut  SFFloat  outputVis
        eventOut  SFVec3f  outputPos
        url "javascript:
          function popisky(value)\{
            if (value)\{
              outputVis = 0.0;
              outputPos[0] = 0.0;
              outputPos[1] = 0.0;
              outputPos[2] = 8.0;
            \}else\{
              outputVis = 0.5;
              outputPos[0] = 0.0;
              outputPos[1] = 0.0;
              outputPos[2] = 1.0;
            \}
          \}
        "
      \}
      Billboard\{
        axisOfRotation 0. 0. 0.
        children[
          \klic{DEF} pozice Transform\{
            translation .0 .0 1. \koment{# -fRad/2 -fRad/2 fRad}
            children[
              Shape\{      \koment{#znacka prvku}
                geometry Text \{
                  string \klic{IS} sSig
                  fontStyle FontStyle\{
                    justify ["MIDDLE","MIDDLE"]
                  \}
                \}
                appearance
                  Appearance\{
                    material \klic{DEF} vzhled Material \{
                      diffuseColor 1. .0 .0  \koment{#barva popisku}
                      transparency 1.
                    \}
                  \}
              \}
              Transform\{  \koment{#seznam lokantu}
                translation .0 -1. .0
                children[
                  Shape\{
                    geometry Text \{
                      string \klic{IS} sLoc
                      fontStyle FontStyle\{
                        justify ["BEGIN","TOP"]
                        family ["Palatino Linotype" ]
                      \}
                    \}
                    appearance
                      Appearance\{
                        material \klic{DEF} vzhledLoc Material \{
                          diffuseColor 1. 1. .0  \koment{#barva popisku}
                          transparency 1.
                        \}
                      \}
                  \}
                ]
              \}
            ]
          \}
          Shape\{
            geometry Sphere \{
              radius .5              \koment{#polomer vnitrni casti}
            \}
            appearance
              Appearance\{
                material Material \{
                  diffuseColor \klic{IS} dCol
                  transparency 0.    \koment{#vnitrni cast - viditelnost}
                \}
              \}
          \}
          Shape\{
            geometry Sphere \{
              radius \klic{IS} fRad
            \}
            appearance
              Appearance\{
                material Material \{
                  diffuseColor \klic{IS} dCol
                  transparency .5    \koment{#vnejsi cast - viditelnost}
                \}
              \}
          \}
          \klic{DEF} tl TouchSensor\{  \koment{#atom bude reagovat na stisk tl.}
          \}
        ]
      \}
    ]
  \}
  \klic{ROUTE} tl.isActive \klic{TO} java.popisky
  \klic{ROUTE} java.outputVis \klic{TO} vzhled.transparency
  \klic{ROUTE} java.outputVis \klic{TO} vzhledLoc.transparency
  \klic{ROUTE} java.outputPos \klic{TO} pozice.translation
\}

\koment{########################### vazby - vzory ###########################}
\klic{PROTO} Spoj[
    field SFColor dCol 1.0 1.0 1.0
    field SFFloat fLen 2.
    field SFVec3f fPos 0 1 0.  #0 fLen/2 0
    field SFFloat fRad .5
]\{
  Transform\{
    translation \klic{IS} fPos
    children[
      Shape\{
        geometry Cylinder \{
          height \klic{IS} fLen
          radius \klic{IS} fRad
        \}
        appearance
        Appearance\{
          material Material \{
            diffuseColor \klic{IS} dCol
            transparency 0
          \}
        \}
      \}
    ]
  \}
\}

\klic{PROTO} Vazba1 [
    field SFFloat fDist 2.
    field SFVec3f fPoz  0 1 0.
]\{
  Spoj \{
    dCol 1.0 1.0 1.0
    fLen \klic{IS} fDist
    fPos \klic{IS} fPoz
  \}
\}

\klic{PROTO} Vazba2 [
    field SFFloat fDist 2.
    field SFVec3f fPoz  0 1 0.
]\{
  Transform\{
    children[
      Transform\{
        translation .16666666 0 0
        children Spoj \{
          dCol 0 1 1
          fRad .33333333
          fLen \klic{IS} fDist
          fPos \klic{IS} fPoz
        \}
      \}
      Transform\{
        translation -.16666666 0 0
        children Spoj \{
          dCol 0 1 1
          fRad .33333333
          fLen \klic{IS} fDist
          fPos \klic{IS} fPoz
        \}
      \}
    ]
  \}
\}

\komentH{# PROTO Vazba3 - obdobne}

\koment{#####################################################################}
\end{kodblok}

