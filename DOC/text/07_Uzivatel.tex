\chapter{Uživatelská dokumentace}
Program \todo{CheNom} je vytvořen za účelem vizualizace organických molekul
podle jejich jména. Zdrojové kódy byly psány tak, aby bylo možné program
provozovat jak pod operačními systémy \todo{Windows (98+)}, tak pod těmi
z~rodiny \todo{Linux}.

Tato kapitola by měla seznámit čtenáře s~možnostmi programu a jejich využíváním
od instalace přes běžné použití až po pokročilou konfiguraci.

\section{Instalace}
K~provozování programu je nutné mít nainstalovaný VRML prohlížeč, a to nejlépe
integrovaný do internetového prohlížeče, který umí otevřít soubor obsahující
v~názvu znak \%. To umožňuje použít WWW rozhraní programu. Pokud se jedná o
standalone aplikaci, je možné program používat z~příkazové řádky.

\subsection{\todo{Windows}}
\begin{enumerate}
	\item Nainstalujte internetový prohlížeč splňující výše uvedenou podmínku
		  (\todo{Internet Explorer} ne) včetně pluginu pro VRML.
	\item Zkopírujte z~přiloženého CD složku \soubor{{\bs}Windows{\bs}CheNom}
		  na místo, kde si přejete program mít.
	\item Pokud jste jako prohlížeč nezvolili \todo{Firefox}, pak v~textovém
		  editoru (např. \todo{Poznámkový blok}) otevřete soubor
		  \soubor{START.bat} a nahraďte slovo \kod{firefox} příkazem pro
		  spuštění vašeho prohlížeče. Změny uložte.
	\item Program se spouští obsaženým skriptem \soubor{START.bat},
		  popř.~jeho konzolovou variantou \soubor{run-console.bat}.
\end{enumerate}

\subsection{\todo{Linux}}
\begin{enumerate}
	\item Nainstalujte VRML plugin pro váš internetový prohlížeč.
	\item Zkopírujte z~přiloženého CD složku \soubor{/Linux/CheNom} na místo,
		  kde si přejete program mít.
	\item Spusťte v~dané složce příkaz \kod{./compile}.
	\item Program se spouští obsaženým skriptem \soubor{START.sh},
		  popř.~jeho konzolovou variantou \soubor{run-console.sh}.
\end{enumerate}

\section{Obsluha přes WWW rozhraní}
Po spuštění serveru je možné přistupovat k~němu přes internetový prohlížeč.
WWW adresa, na níž je služba dostupná je
\uri{http://\textit{pocitac}:\textit{port}/chenom}, tedy pro počítač, na němž je
program spuštěn na portu 80 (standardní spuštění) je to
\stronger{\uri{http://127.0.0.1:80/chenom}}, ale protože port 80 se může vynechat
(std. port pro HTTP protokol), stačí použít
\stronger{\uri{http://127.0.0.1/chenom}}. Pozor, na linuxových systémech smí na
portech nižších než 1024 spouštět služby pouze správce! Pokud jím nejste,
obvykle se postupuje tak, že se použije port 8080.

Obsluha je velmi jednoduchá. Stačí do bílé kolonky ve stránce vyplnit jméno
sloučeniny a stisknout \klavesa{Enter}, příp.~tlačítko \kod{Zobrazit}.
Pro vkládání speciálních znaků a formátování jsou pod touto kolonkou tlačítka.
Modrá tlačítka slouží k~formátování (jmenováno zleva se jedná o kurzívu [\kod{i}],
dolní index [\kod{sub}] a~horní index [\kod{sup}]) a k~vkládání mezery [\kod{\_}].
Přesně tak, jak se tyto symboly používají v~HTML. Pro vkládání malých řeckých
písmen slouží zelená tlačítka [\kod{$\alpha$} -- \kod{$\omega$}].

Po stisknutí odeslání dotazu je tento zpracován a nastat mohou dvě alternativy.
Buď je zadané jméno vyhodnoceno jako chybné, není vygenerován žádný model a je
vrácena stránka obsahující seznam všech vygenerovaných chybových hlášení, nebo
vše proběhne úspěšně, podaří se vygenerovat model zadané molekuly a vrácena je
stránka, která jej obsahuje v~tagu \kod{$<$embed$>$}.

Program se ukončuje kliknutím na červené tlačítko \kod{KONEC} v~pravém horním
rohu. Jedná se o odkaz vedoucí na \stronger{\uri{/STOP}}.

\begin{figure}[h]
	\imgboxB{www}
	\caption{Takto vypadá vygenerovaná stránka}
\end{figure}

\section{Obsluha z~příkazové řádky}
K~obsluze z~příkazové řádky slouží skript \kod{run-console}. Použití je snadné,
stačí zadat do parametru jméno požadované molekuly (může, ale nemusí být
v~uvozovkách, pozor však na meta znaky jako ,,$<$", ,,$>$", ,,$;$", \dots).
Program následně otevře vytvořený model (pokud nebyl název vyhodnocen jako
chybný) v~asociované aplikaci a skript se ukončí.

\section{Manipulace s~modelem}
Kromě běžné manipulace jako je otáčení či přibližování, je možné zjišťovat chemickou
značku atomu a seznam jemu odpovídajících lokantů kliknutím na něj. Navíc
zde tato informace zůstane poloprůhledně zobrazena i po puštění tlačítka myši.
Tato vlastnost může pomoci při orientaci ve složitějších modelech, navíc pomůže
odlišit atomy, které jsou zobrazeny stejnou nebo podobnou barvou.

\section{Konfigurace}
Celý program (ve smyslu každá jeho část) je konfigurovatelný, od změny vzhledu
generovaných modelů až po provedení rozhraní.

\subsection{Databáze názvů}
Databázi názvů molekul, které je program schopný interpretovat, lze snadno a do
libovolné míry rozšiřovat. Nadměrným rozšiřováním se však významně prodlužuje
doba potřebná na vytvoření derivačního stromu a s~ní i celková doba běhu
programu. Při rozšiřování je také potřeba postupovat opatrně a dávat pozor, aby
v~gramatice nevznikl konflikt. Pro tyto účely je dobré vytvořit si sadu názvů
se správnými výsledky a po každé významnější změně spustit skript, který odhalí
případnou neshodu. Další možností je použít připravený skript, který převede
soubor gramatiky do formátu používaného programem bison. Příkaz
\kod{bison -n -r solved gramatika.y} pak vytvoří nepotřebný soubor
\soubor{gramatika.tab.c} a soubor \soubor{gramatika.output}, kde je mimo jiné
popsán stavový automat přijímající danou gramatiku, ale hlavně seznam všech
konfliktů. To usnadňuje jejich hledání a~navíc na rozdíl od testů zaručuje
nalezení všech. Naopak vzniká opačný problém -- nalezení ,,falešných" konfliktů,
které ve skutečnosti žádnými konflikty nejsou (bison totiž umí pracovat pouze
s~gramatikami ze třídy LALR(1), ale gramatika \todo{chenomu} je bezkontextová
(BKJ), což je výrazná nadmnožina [viz {\figurename}~\ref{pic:gramatiky}).

\begin{figure}[h]
	\imgbox{gramatiky}
	\caption{\label{pic:gramatiky}Hierarchie tříd gramatik \cite{Yaghob:pp}}
\end{figure}

Vlastní rozšiřování je snadné. Je-li přidán nový terminál, je potřeba ho
zařadit do seznamu v~souboru \soubor{terminaly.gra} a to na správnou pozici.
Každému terminálu je přiřazeno číslo a to buď jeho zapsáním za něj odděleně
sekvencí bílých znaků, nebo je to číslo o jedna větší než jaké bylo použito na
předchozím řádku. Terminál musí být uveden v~apostrofech, jinak je řádek
považován za komentář (nesmí obsahovat mezeru), ale případné uvedené číslo je
použito.

Správnou pozicí nového terminálu se myslí taková, kdy zůstanou zachovány
příslušnosti k~jednotlivým intervalům definovaným v~souboru
\soubor{konstanty.txt}. Ten vypadá tak, že je na každém řádku definován jeden
interval. Definice má tvar dolní mez, libovolný znak, horní mez, sekvence bílých
znaků (SBZ), pomlčka, SBZ, jméno intervalu, SBZ, popis bez mezer. Pokud je na
místě pomlčky jiný znak, předpokládá se, že již následuje jen SBZ a řádek je
ignorován.

Nejobtížnějším krokem rozšiřování gramatiky je přidání nového pravidla. To se
provádí do souboru \soubor{gramatika.gra}. Jeho syntax je uvedena
v~Pří\-lo\-ze~\ref{priloha:databaze}.

\subsection{Skripty}
Spouštěcí skripty lze jednoduše modifikovat pro specifické potřeby uživatele
pouhým odkomentováním/zakomentováním příslušného řádku. Bližší informace
o~možnostech jsou obsaženy přímo v~komentářích jednotlivých skriptů
(viz~{\appendixname}~\ref{priloha:skripty}).

\subsection{WWW rozhraní}
Celý vzhled WWW stránky použité v~tomto rozhraní lze změnit v~souboru
\soubor{index.htm}. Je pouze nutné zachovat několik následujících pravidel:
\begin{itemize}
	\item Nejsou vkládány žádné soubory (obrázky, videa, skripty, styly, \dots).
	\item Všechna místa, kde má být umístěno jméno zobrazené molekuly, jsou
		  označena řetězcem \kod{\$NAZEV\$} (popř. řetězcem \kod{\$NAZEV*\$},
		  pokud nemají být interpretovány HTML entity, což je vhodné k~předvyplnění
		  aktuálního názvu do vstupního pole pro možnost jeho opravy).
	\item Všechna místa, kde má být vložen výsledek zpracování názvu molekuly,
		  jsou označena řetězcem \kod{\$TELO\$}.
	\item Formulář pro odeslání dotazu \dots
		  \begin{itemize}
				\item je odeslán metodou GET.
				\item má atribut action nastaven na \uri{/chenom}.
				\item obsahuje jediný prvek (kromě tlačítka \kod{submit}) --
					  položku jménem \kod{mol}, v~níž je vyplněn název
					  požadované molekuly.
		  \end{itemize}
	\item Formátování lze provézt pomocí CSS, použity jsou dvě třídy:
		  \begin{itemize}
				\item \kod{error} pro chybový výstup (element \kod{div}),
				\item \kod{vystup} pro okno molekuly (element \kod{embed}).
		  \end{itemize}
\end{itemize}

\subsection{Vzhled modelu}
Různé součásti vzhledu vytvořených modelů jsou definovány na různých místech.
Pokud se má změna vzhledu projevit i v~modelech, které již byly vytvořeny (při
zapnutém cachování [viz {\appendixname}~\ref{priloha:skripty}]), je nutné buď
změnu provést i v~nich, nebo (což je obvykle jednodušší) smazat obsah složky
\soubor{outs} a modely se vytvoří znovu při příštím požadavku na ně již se
správným vzhledem.

Definice vlastností jednotlivých atomů (barva a poloměr) jsou umístěny v~souboru
\soubor{prvky.gra}. Jedná se o 5. (poloměr) a 7. (barva v~HTML formátu) sloupec.
Sloupce jsou odděleny pomocí SBZ. Pokud řádek nezačíná dvojicí znaků ,,$>>$"
(těsně předcházející prvnímu sloupci), je považován za komentář.

Všechny ostatní vlastnosti týkající se vzhledu jsou umístěny v~souboru
\soubor{sablona.wrl}. Jedná se o hlavičku výsledného VRML dokumentu -- modelu.
Jsou zde definovány případné Viewpointy, základní vzhled atomů a vazeb (jména
těchto prototypů a jejich atributy musí být zachovány).

\begin{figure}[h]
	\imgbox{sablona}
	\caption{Hierarchie prototypů v~dokumentu modelu}
\end{figure}

