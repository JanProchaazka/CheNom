\chapter{Technická realizace -- rozhraní}
V~předchozí kapitole je popsáno jádro celého programu. To ovšem pracuje pouze na
příkazové řádce, což sice pro vstup nevadí, ale výstup, tj. 3D model molekuly,
je potřeba nějak interpretovat. Implementována jsou dvě řešení.

\section{run-console}
Jedná se o~velice jednoduchý skript, který očekává jako parametr jméno
požadované molekuly. Zjistí, zda již model existuje ve složce \soubor{outs}.
Pokud ne, je za pomoci jádra na tomto místě vytvořen. Za situace, že se toto
podařilo, je vytvořený model otevřen definované aplikaci (schopné zobrazit
soubory VRML). Defaultně je tímto programem internetový prohlížeč.

\section{HTTP server}
Ačkoliv se jedná spíše o koncový modul celého programu, byl pro vyšší
variabilitu implementován jako samostatný program. Jedná se o malý a co do
funkčnosti velice omezený HTTP server~\cite{Berners:rfc1945}.

Při startu je načtena šablona stránky (tj. obsah souboru \soubor{index.htm},
kde v~místech pro nadpis je text \kod{\$NAZEV\$}, pro nadpis s~neinterpretovanými
HTML entitami \kod{\$NAZEV*\$} a pro tělo \kod{\$TELO\$}). Poté server začne
poslouchat požadavky přicházející na port, jehož číslo bylo zadáno jako parametr
(doporučeno je číslo běžně užívané pro HTTP servery -- 80). Kladně jsou vyřízeny
pouze následující požadavky:
\begin{itemize}
	\item \kod{GET /chenom},
	\item \kod{GET /chenom?mol=nazev\_molekuly} -- pokusí se vytvořit model
		  molekuly s~daným názvem (pokud se to podaří, vrátí stránku, která model
		  obsahuje, jinak se na stránce zobrazí seznam vygenerovaných chy\-bo\-vých
		  hlášení),
	\item \kod{GET /outs/nazev\_molekuly.wrl} -- vrátí model, který existuje,
	\item \kod{GET /STOP} -- ukončí server,
	\item \kod{GET /RESET} -- způsobí znovunačtení šablony ze souboru
		  \soubor{index.htm}.
\end{itemize}
V~ostatních případech je vrácena stránka s~chybovou hláškou.

V~druhé variantě dotazu je pro zajištění existence dotazovaného modelu molekuly
použit skript \soubor{run}, který volá jádro \modul{chenom} s~volitelným
parametrem \kod{-e}. Ten způsobí, že je na vstupu očekáván oescapovaný (pomocí
,,htmlencode") název. Výstup je poté uložen do stejnojmenného souboru do složky
\soubor{outs}. Chyby jsou ukládány do souboru \soubor{error.log}, jehož obsah je
v~případě neúspěchu generování zobrazen uživateli.

Server je navrhován jako localhostová aplikace a jako taková nemá velké nároky
na provoz. Navíc zpracovávání dotazu může trvat i několik minut při 100\% zátěži
procesoru. Z~těchto důvodů bylo při implementaci přistoupeno k~jednovláknové
variantě.


